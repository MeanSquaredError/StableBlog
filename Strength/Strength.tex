\documentclass[12pt]{article}
\usepackage{graphicx} % Required for inserting images
\usepackage[parfill]{parskip}
\usepackage{verbatimbox}
\usepackage{fancyvrb}
\usepackage{exercise}
\usepackage{hyperref}

% Palatino for main text and math
\usepackage[osf,sc]{mathpazo}

% Helvetica for sans serif
% (scaled to match size of Palatino)
\usepackage[scaled=0.90]{helvet}

%
%
\makeatletter
\@ifundefined{lhs2tex.lhs2tex.sty.read}%
  {\@namedef{lhs2tex.lhs2tex.sty.read}{}%
   \newcommand\SkipToFmtEnd{}%
   \newcommand\EndFmtInput{}%
   \long\def\SkipToFmtEnd#1\EndFmtInput{}%
  }\SkipToFmtEnd

\newcommand\ReadOnlyOnce[1]{\@ifundefined{#1}{\@namedef{#1}{}}\SkipToFmtEnd}
\usepackage{amstext}
\usepackage{amssymb}
\usepackage{stmaryrd}
\DeclareFontFamily{OT1}{cmtex}{}
\DeclareFontShape{OT1}{cmtex}{m}{n}
  {<5><6><7><8>cmtex8
   <9>cmtex9
   <10><10.95><12><14.4><17.28><20.74><24.88>cmtex10}{}
\DeclareFontShape{OT1}{cmtex}{m}{it}
  {<-> ssub * cmtt/m/it}{}
\newcommand{\texfamily}{\fontfamily{cmtex}\selectfont}
\DeclareFontShape{OT1}{cmtt}{bx}{n}
  {<5><6><7><8>cmtt8
   <9>cmbtt9
   <10><10.95><12><14.4><17.28><20.74><24.88>cmbtt10}{}
\DeclareFontShape{OT1}{cmtex}{bx}{n}
  {<-> ssub * cmtt/bx/n}{}
\newcommand{\tex}[1]{\text{\texfamily#1}}	% NEU

\newcommand{\Sp}{\hskip.33334em\relax}
\newcommand{\NB}{\textbf{NB}}
\newcommand{\Todo}[1]{$\langle$\textbf{To do:}~#1$\rangle$}

\EndFmtInput
\makeatother
%

% Bera Mono for monospaced
% (scaled to match size of Palatino)
% \usepackage[scaled=0.85]{beramono}

\title{What does it mean for a monad to be strong?}
\author{Dan Piponi}
\date{June 2024 (originally August 2023)}

\begin{document}

\maketitle

This is something I put on github years ago but I probably should have put it here.

Here's an elementary example of the use of the list monad:
\begin{tabbing}\ttfamily
~~~test1~\char61{}~do\\
\ttfamily ~~~~~x~\char60{}\char45{}~\char91{}1\char44{}~2\char93{}\\
\ttfamily ~~~~~y~\char60{}\char45{}~\char91{}x\char44{}~10\char42{}x\char93{}\\
\ttfamily ~~~~~\char91{}x\char42{}y\char93{}
\end{tabbing}
We can desugar this to:
\begin{tabbing}\ttfamily
~~~test2~\char61{}~\char91{}1\char44{}~2\char93{}~\char62{}\char62{}\char61{}~\char92{}x~\char45{}\char62{}~\char91{}x\char44{}~10\char42{}x\char93{}~\char62{}\char62{}\char61{}~\char92{}y~\char45{}\char62{}~\char91{}x\char42{}y\char93{}
\end{tabbing}
It looks like we start with a list and then apply a sequence (of length 2) of functions to it using
bind \text{\ttfamily \char40{}\char62{}\char62{}\char61{}\char41{}}. This is probably why some people call monads workflows and why the comparison has been
made with Unix \href{http://okmij.org/ftp/Computation/monadic-shell.html}{pipes}.

But looks can be deceptive. The operator \text{\ttfamily \char40{}\char62{}\char62{}\char61{}\char41{}} is right associative and \text{\ttfamily test2} is the same as \text{\ttfamily test3}:
\begin{tabbing}\ttfamily
~~~test3~\char61{}~\char91{}1\char44{}~2\char93{}~\char62{}\char62{}\char61{}~\char40{}\char92{}x~\char45{}\char62{}~\char91{}x\char44{}~10\char42{}x\char93{}~\char62{}\char62{}\char61{}~\char92{}y~\char45{}\char62{}~\char91{}x\char42{}y\char93{}\char41{}
\end{tabbing}
You can try to parenthesise the other way:
\begin{tabbing}\ttfamily
~~~\char45{}\char45{}~test4~\char61{}~\char40{}\char91{}1\char44{}~2\char93{}~\char62{}\char62{}\char61{}~\char92{}x~\char45{}\char62{}~\char91{}x\char44{}~10\char42{}x\char93{}\char41{}~\char62{}\char62{}\char61{}~\char92{}y~\char45{}\char62{}~\char91{}x\char42{}y\char93{}
\end{tabbing}
We get a "Variable not in scope: x" error. So \text{\ttfamily test1} doesn't directly fit the workflow model. When
people give examples of how workflow style things can be seen as monads they sometimes use
examples where later functions don't refer to variables defined earlier. For example at the link I
gave above the line \text{\ttfamily m~\char62{}\char62{}\char61{}~\char92{}x~\char45{}\char62{}~\char40{}n~\char62{}\char62{}\char61{}~\char92{}y~\char45{}\char62{}~o\char41{}} is transformed to \text{\ttfamily \char40{}m~\char62{}\char62{}\char61{}~\char92{}x~\char45{}\char62{}~n\char41{}~\char62{}\char62{}\char61{}~\char92{}y~\char45{}\char62{}~o} which
only works if \text{\ttfamily o} makes no mention of \text{\ttfamily x}. I found similar things to be true in a number of tutorials,
especially the ones that emphasise the Kleisli category view of things.

But we can always "reassociate" to the left with a little bit of extra work.  The catch is that the
function above defined by \text{\ttfamily \char92{}y~\char45{}\char62{}~\char46{}\char46{}\char46{}} "captures" \text{\ttfamily x} from its environment.  So it's not just one
function, it's a family of functions parameterised by \text{\ttfamily x}.  We can fix this by making the dependence
on \text{\ttfamily x} explicit.  We can then pull the inner function out as it's no longer implicitly dependent on
its immediate context.  When compilers do this it's called \href{https://en.wikipedia.org/wiki/Lambda_lifting}{lambda lifting}.

Define (the weirdly named function) \text{\ttfamily strength} by
\begin{tabbing}\ttfamily
~~~strength~\char58{}\char58{}~Monad~m~\char61{}\char62{}~\char40{}x\char44{}~m~y\char41{}~\char45{}\char62{}~m~\char40{}x\char44{}~y\char41{}\\
\ttfamily ~~~strength~\char40{}x\char44{}~my\char41{}~\char61{}~do\\
\ttfamily ~~~~~y~\char60{}\char45{}~my\\
\ttfamily ~~~~~return~\char40{}x\char44{}~y\char41{}
\end{tabbing}
It allows us to smuggle \text{\ttfamily x} "into the monad".

And now we can rewrite \text{\ttfamily test1}, parenthesising to the left:
\begin{tabbing}\ttfamily
~~~test5~\char61{}~\char40{}\char91{}1\char44{}~2\char93{}~\char62{}\char62{}\char61{}~\char92{}x~\char45{}\char62{}~strength~\char40{}x\char44{}~\char91{}x\char44{}~10\char42{}x\char93{}\char41{}\char41{}\\
\ttfamily ~~~~~~~~~~~~~~~~~~~\char62{}\char62{}\char61{}~\char92{}\char40{}x\char44{}~y\char41{}~\char45{}\char62{}~\char91{}x\char42{}y\char93{}
\end{tabbing}
This is much more like a workflow. Using \text{\ttfamily strength} we can rewrite any (monadic) \text{\ttfamily do} expression as a
left-to-right workflow, with the cost of having to throw in some applications of \text{\ttfamily strength} to carry
along all of the captured variables. It's also using a composition of arrows in the Kleisli category.

A monad with a strength function is called a strong monad. Clearly all Haskell monads are strong as I
wrote \text{\ttfamily strength} to work with any Haskell monad. But not all monads in category theory are strong.
It's a sort of hidden feature of Haskell (and the category Set) that we tend not to refer to explicitly.
It could be said that we're implicitly using strength whenever we refer to earlier variables in our
\text{\ttfamily do} expressions.

See also \href{https://ncatlab.org/nlab/show/strong+monad}{nlab}.
\begin{tabbing}\ttfamily
~~~main~\char61{}~do\\
\ttfamily ~~~~~print~test1\\
\ttfamily ~~~~~print~test2\\
\ttfamily ~~~~~print~test3\\
\ttfamily ~~~~~\char45{}\char45{}~print~test4\\
\ttfamily ~~~~~print~test5
\end{tabbing}
\end{document}
